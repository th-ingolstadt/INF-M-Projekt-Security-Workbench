%!TEX root = ../document.tex
\chapter{Einleitung}
Dieses Dokument beschreibt die Verwendung der Broken Web Application. Diese ist Teil der Security Workbench, die seit
dem Sommersemester 2016 als studentische Projektarbeit im Rahmen des Masterstudiengangs
Informatik an der TH Ingolstadt entwickelt wird. Die Workbench
erklärt und veranschaulicht verschiedene Angriffe und Szenarien aus dem Bereich
der Netzwerksicherheit. Dies betrifft unter anderem Spoofing, Denial-of-Service
und Angriffe auf die WLAN Infrastruktur. \\ 
Im Wintersemester 16/17 wurde das Projekt der Security Workbench um eine Broken Web Application erweitert, die dem Einsteiger erlaubt gängige Angriffsarten wie Cross-Site-Scripting, SQL-Injection oder Broken Authentication und Session Management anhand von Beispielübungen kennen zu lernen. \\ 
Nach einer Erläuterung wie die Broken Web Application unter einer virtuellen Maschine vorbereitet werden muss zeigen die folgenden Kapitel auf, wie die einzelnen Tutorials zu den Angriffsarten durchgeführt werden können. 

\section{Aktualisierung}

Die Broken Web Application liegt unter dem öffentlichen Git-Repository der Security Workbench \\ \url{https://github.com/th-ingolstadt/INF-M-Projekt-Security-Workbench} vor. Hierzu ist der Source Code der Webanwendung im Ordner \textit{SecWorkbench} wiederzufinden. Eine Aktualisierung per git kann auf der Kommandozeile wie folgt durchgeführt werden.

\begin{lstlisting}
> cd INF-M-Projekt-Security-Workbench
> git pull
\end{lstlisting}

\section{Weiterentwicklung} 
Die Webanwendung basiert auf einem Projekt, bei dem die Serverseite mittels PHP programmiert ist. Clientseitig ist besteht die Applikation aus HTML-Seiten, CSS für das Design und JavaScript für die Logik. Um das Design zu vereinfachen wird mit dem Framework \textit{Bootstrap} entwickelt worden, das dem Entwickler eine Reihe von vordefinierten Styles mit an die Hand gibt (\url{https://getbootstrap.com/}). Darüber hinaus liegt der Weboberfläche das Dashboard \textit{AdminLTE} zugrunde, dass weitere Style-Vereinfachungen bietet (\url{https://adminlte.io/themes/AdminLTE/index2.html}). Besonders in Bezug auf künftige Entwicklungen, können diese Frameworks weiterhin verwendet und so die Weboberfläche einheitlich gestaltet werden.  



