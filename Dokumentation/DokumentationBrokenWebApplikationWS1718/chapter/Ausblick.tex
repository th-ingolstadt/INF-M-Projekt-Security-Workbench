%!TEX root = ../document.tex
\chapter{Ausblick}
\label{Ausblick}
Das nachfolgende Kapitel beschreibt kurz welche Erweiterungsmöglichkeiten für die Broken Web Application möglich sind und welche Verbesserungen angestrebt werden können.

\section{Erweiterungsmöglichkeiten}
\label{sec:erweiterungsmöglichkeiten}
Die Broken Web Application weist bisher die Angriffsarten Broken Session Management, SQLInjection, Bufferoverflow, Cross-Site-Scripting und einen Login-Parcour auf. Prinzipiell kann die Webanwendung um weitere Angriffsarten erweitert und die bisherigen Tutorials ausgebaut werden. Zur Orientierung kann hierzu die OWASP Top 10 sowie deren Demo-Webanwendungen herangezogen werden. (\url{https://www.owasp.org/index.php/Main_Page}). \\
Des Weiteren ist der Betrieb der Webanwendung nur bedingt auf den RaspberryPi-Bildschirmen der Security Workbench ausgelegt, da diese parallel entwickelt wurden. Im Zuge dessen kann die Broken Web Application auf ein Responsive Webdesign erweitert werden um auf allen Gerätbildschirmen lauffähig zu sein. Das bereits verwendete Framework \textit{Bootstrap} erleichtert hierzu die Verbesserungen.