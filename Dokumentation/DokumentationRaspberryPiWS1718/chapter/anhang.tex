\chapter{Anhang}

\section{Docker-Merkblatt}
\label{sec:docker_merkblatt}
Dieses Kapitel ist an Entwickler gerichtet, die mit Docker arbeiten.
Zur Hilfestellung werden hier einige nützliche Kommandos aufgeführt.

\subsection*{Allgemein}
\begin{itemize}
	\item docker info -> Gesamtüberblick über die lokale Umgebung
\end{itemize}

\subsection*{Images}
\begin{itemize}
	\item docker images -> Listet alle verfügbaren Images auf
	\item docker rmi <image> -> Löscht ein Image
	\item docker build <Verzeichnis des Dockerfiles> -> Erzeugt ein Image
	\begin{itemize}
		\item[$\triangleright$] -t <image> -> Vergibt dem Image einen Namen
	\end{itemize}
\end{itemize}


\subsection*{Container}

\begin{itemize}
	\item docker ps -> Listet alle laufenden Container auf
	\begin{itemize}
		\item -a -> Listet alle Container auf
	\end{itemize}
	
	\item docker start <container> -> Startet einen Container
	\item docker stop <container> -> Stoppt einen Container
	\item docker logs <container> -> Zeigt Logfile des Containers 
	\item docker run <image> -> Erzeugt \& startet Container
	\begin{itemize}
		\item[$\triangleright$] --name <container> -> Vergibt Containernamen
		\item[$\triangleright$] -p <ip>:<host-port>:<container-port> -> IP-Adresse des Containers und Portmapping zwischen Container und Host (WICHTIG: Dockerfile muss Port freigeben mittels EXPOSE)
		\item[$\triangleright$] -d -> Container wird im Hintergrund gestartet
		\item[$\triangleright$] -v <host-Verzeichnis>:<container-Verzeichnis> -> Bindet ein Volume vom Host in den Container ein
	\end{itemize}
	\item docker exec -it <container> <kommando> -> Führt ein Kommando in einem laufenden Container aus (HINWEIS: Um eine Bash in einem laufenden Container zu öffnen docker exec -it <container> /bin/bash eingeben)
\end{itemize}