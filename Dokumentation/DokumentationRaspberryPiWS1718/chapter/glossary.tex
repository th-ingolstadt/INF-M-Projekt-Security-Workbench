%!TEX root = ../document.tex
\chapter{Fachbegriffe}
Hier werden wichtige Fachbegriffe im Kontext der Security Workbench kurz erklärt, die im späteren Verlauf für die einzelnen Angriffe eine Rolle spielen.
%TODO Eventuell in ein echtes Glossar umbauen

\section{MAC-Adresse}

Die MAC-Adresse -- kurz für \emph{Media Access Control} -- ist eine Hardwareadresse eines Netzwerkadapters, welches eben dieses Adapter im lokalen Netzwerk identifiziert. Jede LAN"=Schnittstelle und jedes WLAN"=Interface benötigt eine eigene MAC-Adresse. Eine solche MAC-Adresse ist sechs Byte lang und wird üblicherweise in hexadezimaler Notation angegeben.
\begin{verbatim}
E8-03-9A-DC-DF-23
\end{verbatim}
Unter Windows kann die eigene MAC-Adresse per \verb|ipconfig -all| bestimmt werden, unter Linux wird hierfür \verb|ifconfig -a| verwendet. Die MAC-Adresse wird pro Gerät üblicherweise vom Hersteller vergeben, daher kann anhand der ersten drei Byte über öffentlich zugängliche Datenbanken\footnote{siehe etwa \url{http://www.macvendorlookup.com/}} ein Rückschluss auf die Firma gezogen werden, welche das Netzwerkgerät produziert hat. Entsprechende Datenbanken ordnen beispielsweise obige MAC der \enquote{Samsung Electronics CO., LTD} zu.

\section{HTTP}
Hypertext Transfer Protocol (HTTP) ist ein zustandsloses Protokoll zur Übertragung von Daten auf Anwendungsschicht über ein Rechnernetz. Der Standard wurde 1991 von der Internet Engineering Task Force (IETF) und dem World Wide Web Consortium (W3C) eingeführt und ist mittlerweile in Version 2.0 (HTTP/2) veröffentlicht. Es wird meist dafür verwendet, Webseiten aus dem Internet in einen Webbrowser zu laden.

Wird ein Link zu einer URL mit dem Beginn \enquote{http://} aufgerufen, wird HTTP genutzt. Als Erstes wird dann verusch den Namen der Website mit Hilfe des DNS-Protokolls in eine IP-Adresse zu übersetzen (weitere Erklärung siehe Kapitel DNS Spoofing). Ist dies nicht möglich, wird über den Standard-Port 80 eine HTTP-GET-Anforderung gesendet. Als Antwort schickt der Web-Server die passende IP-Adresse der angefragten Webseite.

\section{HTTPS}
Hypertext Transfer Protocol Secure (HTTPS) wird zur sicheren Übertragung von Daten auf der Anwendungsschicht über ein Rechnernetz verwendet. Syntaktisch ist es wie HTTP aufgebaut, wird jedoch von eine Verschlüsselung der Daten umgeben. Dazu wird das Secure Socket Layer (SSL) bzw. die Transport Layer Security (TLS) verwendet.

Bei der Benutzung wird vor dem Versenden und Bearbeiten von Nachrichten eine Identifikation und Authentifizierung der Kommunikationspartner durchgeführt. Danach wird ein gemeinsamer Schlüssel ausgetauscht mit dem alle nachfolgenden Nachrichten verschlüsselt werden. Dabei ist der Standard-Port für HTTPS-Nachrichten Port 443.

\section{SSID/ESSID}
Ein Service Set Identifier (SSID), seltener auch ESSID (Extended SSID) bezeichnet, ist ein vom Nutzer frei zu wählender Name eines Services (WLAN-Netz), über das der Service ansprechbar ist. Ein SSID kann bis zu 32 Byte lang sein und entsprechend bis zu 32 ASCII-Zeichen umfassen.
\section{BSSID}
Die Basic Service Set Identification (BSSID) jedes WLAN-Gerätes ist eindeutig. Im Allgemeinfall versteht man unter der BSSID die MAC-Adresse des Gerätes. 
