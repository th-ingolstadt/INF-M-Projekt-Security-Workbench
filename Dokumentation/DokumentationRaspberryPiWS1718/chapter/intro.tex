%!TEX root = ../document.tex
\chapter{Einleitung}

Dieses Dokument beschreibt die Verwendung der Security Workbench, die seit dem Sommersemester\,2016 als studentische Projektarbeit im Rahmen des Masterstudiengangs Informatik an der TH\,Ingolstadt entwickelt wird. Die Workbench erklärt und veranschaulicht verschiedene Angriffe und Szenarien aus dem Bereich der Netzwerksicherheit. Dies betrifft unter anderem Spoofing, Denial-of-Service und Angriffe auf die WLAN Infrastruktur.

Nach einer Erläuterung relevanter Fachbegriffe und der Erklärung zur grundlegenden Einrichtung der Workbench zeigen die folgenden Kapitel auf, wie die einzelnen Angriffe gestartet werden und welche Voraussetzungen für diese gelten.


\section{Rechtliches}

Für die Verwendung der hier zusammengestellten Tools sei ausdrücklich auf das Kapitel \ref{ch:disclaimer} Disclaimer verwiesen.

\section{Aktualisierung}

Die Security Workbench liegt als öffentliches Git-Repository unter der URL \url{https://github.com/th-ingolstadt/INF-M-Projekt-Security-Workbench} vor. Eine Aktualisierung per git kann auf der Kommandozeile wie folgt durchgeführt werden.

\begin{lstlisting}
> cd INF-M-Projekt-Security-Workbench
> git pull
\end{lstlisting}
