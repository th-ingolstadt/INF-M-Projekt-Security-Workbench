%!TEX root = ../document.tex
\chapter{Vorbereitung}
In diesem Kapitel soll erläutert werden, wie das Web-Projekt bereitgestellt und benutzt wird. Hierzu folgt eine kurze Anleitung um einen Host zu installieren, der die Web-Applikation bereitstellt sowie eine Beschreibung mit dem Umgang der bereits installierten virtuellen Maschine.

\section{Installationsanleitung ohne Docker}
\label{secInstall}

Für die Installation der Web-Applikation ohne den Docker-Container zu nutzen, muss zuerst ein Webserver auf dem Linux Host-Gerät installiert werden, wie z. B. der Apache Server (\url{https://httpd.apache.org}). Eine einfach Installation ist mit dem Befehl \colorbox{altgray}{\lstinline|apt-get install apache2|} möglich. Da sich das Installations-Vorgehen mit den Versionen der Drittanbieter-Software ändert, verzichte ich hier auf eine detaillierte Schritt-für-Schritt-Anleitung. Damit das Webprojekt richtig ausgeführt wird, muss in der apache2.conf folgende zwei Zeilen eingefügt werden:\medskip

\begin{itemize}
	\item \bashCommand{RemoveHandler .html .htm}
	\item \bashCommand{AddType application/x-httpd-php .php .htm .html}\medskip
\end{itemize}
	
Diese Konfiguration sorgt dafür, dass HTML-Seiten mit dem PHP-Interpreter ausgeführt werden. Nach der Konfiguration des Apache-Webservers, muss der PHP-Interpreter installiert werden, dies kann ebenfalls mit dem Package-Manager durchgeführt werden. Dazu wird das Kommando \colorbox{altgray}{\lstinline|apt-get install php|} verwendet. Zum Zeitpunkt der Entwicklung des Projekts, wurde die PHP-Version 7.0.19 verwendet. Um die Installation auf Korrektheit zu prüfen, kann der Befehl \colorbox{altgray}{\lstinline|php --version|} benutzt werden. Dieser zeigt die Versionsnummer der PHP-Umgebung, falls die Installation korrekt durchgeführt wurde.\medskip

Um alle Tutorials durchführen zu können muss der Host einen GNU-Debugger (GDB) bereitstellen. Dieser kann leicht mit  \colorbox{altgray}{\lstinline|apt-get install gdb|} bezogen werden. Um den GDB zu testen, kann der Befehl \colorbox{altgray}{\lstinline|gdb --version|} ausgeführt und die Version angezeigt werden.\medskip

Zuletzt muss noch eine MySQL-Datenbank auf dem Host-System installiert werden. Hierfür stehen mehrere Alternative Vorgehensweisen zur Verfügung, siehe dazu \url{https://dev.mysql.com/doc/refman/5.7/en/linux-installation.html}. Anschließend muss man das MySQL-Skript zur Initialisierung ausführen, dies ist mit dem folgenden Kommando möglich\\ \colorbox{altgray}{\lstinline|mysql < PROJEKTROOT/Projekte/Docker/server/initalizeDB.sql|} \medskip

Um nun die Applikation verwenden zu können, müssen zunächst alle Web-Ressourcen aus dem GitHub-Repository in den von Apache erwarteten Pfad kopieren. Dazu muss das Repository auf den Server kopiert werden, dies kann mit dem Befehl \colorbox{altgray}{\lstinline|git clone https://github.com/th-ingolstadt/INF-M-Projekt-Security-Workbench.git|} erreicht werden. Danach muss der Inhalt des Pfads PROJEKTROOT/Projekte/SecWorkbench/html in das Root-Verzeichnis des Apache-Servers kopiert werden. Dieser ist standardmäßig /var/www/html. Des Weiteren müssen die Rechte des Apache-Users auf das Verzeichnis angepasst werden, dies ist mit dem folgenden Befehl möglich \colorbox{altgray}{\lstinline|sudo chown -R www-data:www-data /var/www/html/|}.

\section{Installationsanleitung mit Docker}

Die Installation der Web-Anwendung mithilfe des bereits konfigurierten Docker-Containers benötigt zuerst die Installation von Docker. Dies kann über den Package-Manager mit \colorbox{altgray}{\lstinline|apt-get install Docker|} durchgeführt werden. Danach sollte, wie in \ref{secInstall} beschrieben, das GitHub-Repository geklont werden. Dort befindet sich der Ordner PROJEKTROOT/Projekte/Docker/server. Dieses Verzeichnis muss nun auf den Desktop des Root-Users kopiert werden. Aus dem eben kopierten Ordner muss das StartUp.sh ebenfalls in das Desktop Verzeichnis verschoben werden. 

\section{Nutzung der virtuellen Maschine}


