%!TEX root = ../document.tex
\chapter{Cross-Site-Scripting}
\label{XSS}
Cross Site Scripting
\section{Erklärung}
blabla bla bla \\ 

\subsection{Angriffsvarianten}
Grundlegend gibt es drei verschiedene Arten von Cross-Site-Scripting Angriffen: \colorbox{altgray}{\lstinline|Reflected XSS, Stored XSS, DOM-Based XSS|}. Um diese möglichst einfach zu erklären, wird im Folgenden der JavaScript Code \colorbox{altgray}{\lstinline|alert("42")|} als Beispiel verwendet. Hierbei kann aber auch jeder andere beliebige JavaScript Code eingeschleust werden. Je nachdem was das Ziel des Angreifers ist, werden bspw. mit \colorbox{altgray}{\lstinline|alert(document.cookie)|}, die gespeicherten Cookies der Webseite auslesen. \\ 
\subsubsection*{Reflected XSS}
\subsubsection*{Stored XSS}
Im Vergleich zum reflektierenden XSS-Angriff unterscheidet sich der Stored XSS (auch persistent/ persistentes XSS) dadurch, dass der Schadcode auf dem Webserver gespeichert wird. Besonders problematisch dabei ist, dass bei jeder Clientanfrage der Schadcode automatisch ausgeliefert und ausgeführt wird. \\ 
\subsubsection*{DOM-Based XSS}
Die dritte Angriffsart von Cross-Site-Scripting bezieht sich ausschließlich auf statische HTML-Seite mit JavaScript Unterstützung lässt dabei den Server außen vor. 


\section{Ablauf}
\subsection{Reflected Cross-Site-Scripting}
\subsection{Stored Cross-Site-Scripting}
\subsection{DOM-Based Cross-Site-Scripting}

\section{Gegenmaßnahmen}

