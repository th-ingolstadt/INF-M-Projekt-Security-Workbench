%!TEX root = ../document.tex
\chapter{Vorbereitung}

\section{Installationsanleitung}
\begin{itemize}
	\item Detaillierte Installationsanleitung mit vielen Screenshots.
	\item Liste der grundsätzlich benötigten Hard- und Software (PC/Notebook, keine alten Intel Pentium CPUs, Kali-VM inkl. VirtualBox bzw. Kali auf bootable USB inkl. Disk Imager etc.)
	\item Ggf. notwendige Grundkonfigurationen in Kali
\end{itemize}

\section{Einführung in das Arbeiten mit Linux}
Es ist davon auszugehen, dass Studenten der unteren Semester keine Erfahrungen im Umgang mit der Konsole, geschweige denn Linux haben. Hier könnte daher eine kurze Einführung in die Arbeitsweise von Linux und ein Cheat-Sheet mit den wichtigsten Befehlen stehen (ggf. Link auf Website)

\section{Weitere Konfigurationen}
Hier können mehrfach benötigte Konfigurationen, Software- und Hardware-Voraussetzungen, die für die einzelnen Angriffsszenarien (z.B. ARP-Spoofing und DNS-Spoofing) notwendig sind, ausgelagert erklärt werden. Damit kann der Anwender jedes Szenario für sich isoliert ausführen, ohne sich die darum liegenden Themen ansehen zu müssen. Z.B.: Tunneln der Netzwerkadapter (benötigt in ARP-Spoofing und DNS-Spoofing)
