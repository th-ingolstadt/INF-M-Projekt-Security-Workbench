%!TEX root = ../document.tex
\chapter{Disclaimer}
\label{ch:disclaimer}

Das vorliegende Dokument und das zugehörige Tool \enquote{Security Workbench} sind im Rahmen eines Projektes des Masterstudiengangs Informatik an der Technischen Hochschule Ingolstadt (THI) im Wintersemester 2016/17 entstanden. Sinn und Zweck der Security Workbench ist es, interessierten Studierenden das Thema IT-Security näher zu bringen. Alle hier gezeigten Tutorials sind ausschließlich für den Einsatz innerhalb einer eigens dafür geschaffenen Umgebung (z.B. dediziertes WLAN zum Durchspielen der Angriffsszenarien) mit der Zustimmung aller Beteiligten (sowohl Angreifer als auch Angegriffene) gedacht.

Der Missbrauch der zur Verfügung gestellten Informationen und Tutorials für kriminelle Handlungen kann strafrechtliche Folgen nach sich ziehen. Strafrechtliche Grundlagen sind hierbei u.a.:
\begin{itemize}
	\item§202 StGB – Ausspähen von Daten
	\item§263 StGB – Computerbetrug
	\item§269 StGB – Fälschung beweiserheblicher Daten
	\item§270 StGB – Täuschung im Rechtsverkehr bei DV
	\item§§ 271, 274, 348 StGB – Falschbeurkundung/Urkundenunterdrückung im Zusammenhang mit DV
	\item§303a StGB – Datenveränderung
	\item§303b StGB – Computersabotage
\end{itemize}

Haftungsansprüche gegen die Autoren oder die THI im Falle der missbräuchlichen Verwendung der Informationen und des Tutorials sind ausgeschlossen. Die Autoren und die THI distanzieren sich ausdrücklich von der Verwendung der Informationen und des Tutorials für kriminelle Handlungen.

Die Autoren und die THI übernimmt keinerlei Gewähr für die Aktualität, Korrektheit, Vollständigkeit oder Qualität der bereitgestellten Informationen. Haftungsansprüche gegen die Autoren oder die THI, welche sich auf Schäden materieller oder ideeller Art beziehen, die durch die Nutzung oder Nichtnutzung der dargebotenen Informationen und Tutorials verursacht wurden, sind grundsätzlich ausgeschlossen. Die Autoren behalten es sich ausdrücklich vor, Teile der Dokumentation bzw. des Tutorials oder das gesamte Angebot ohne gesonderte Ankündigung zu verändern, zu ergänzen, zu löschen oder die Veröffentlichung zeitweise oder endgültig einzustellen.

Bei direkten oder indirekten Verweisen auf fremde Quellen und Internetseiten, die außerhalb des Verantwortungsbereichs der Autoren liegen, würde eine Haftungsverpflichtung ausschließlich in dem Fall in Kraft treten, in dem die Autoren von den Inhalten Kenntnis haben und es ihnen technisch möglich und zumutbar wäre, die Nutzung im Falle rechtswidriger Inhalte zu verhindern. Die Autoren erklären daher ausdrücklich, dass zum Zeitpunkt der Linksetzung die entsprechenden verlinkten Seiten frei von illegalen Inhalten waren. Die Autoren haben keinerlei Einfluss auf die aktuelle und zukünftige Gestaltung und auf die Inhalte der verknüpften Quellen und Seiten. Deshalb distanzieren sie sich hiermit ausdrücklich von allen Inhalten aller verknüpften Quellen und Seiten, die nach der Verknüpfung verändert wurden. Für illegale, fehlerhafte oder unvollständige Inhalte und insbesondere für Schäden, die aus der Nutzung oder Nichtnutzung solcherart dargebotener Informationen entstehen, haftet allein der Anbieter der Seite, auf welche verwiesen wurde, nicht derjenige, der über Links auf die jeweilige Veröffentlichung lediglich verweist.
